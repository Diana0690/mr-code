\documentclass[a4paper,12pt]{article} % ??\%
\usepackage{color}
\usepackage{epsfig}
\usepackage{amsfonts}
\usepackage{amsmath}
\usepackage{amssymb, amsbsy}
\usepackage{mathtools}
\usepackage{chngpage}
\usepackage{rotating}
\usepackage{multirow}
\usepackage{setspace}
\usepackage{url}
\usepackage{listings}
\lstset{
basicstyle=\small\ttfamily,
columns=flexible,
breaklines=true,
breakatwhitespace=true,
showstringspaces=false,
showspaces=false,
language=R
}
\DeclareMathOperator{\se}{se}
\DeclareMathOperator{\cov}{cov}
\DeclareMathOperator{\cor}{cor}
\DeclareMathOperator{\Cor}{Cor}
\DeclareMathOperator{\var}{var}
\DeclareMathOperator{\sgn}{sgn}
\newcommand{\slfrac}[2]{\left.#1\middle/#2\right.}
\addtolength{\hoffset}{-0.8cm}
\addtolength{\textwidth}{1.6cm}
\addtolength{\voffset}{-1cm}
\addtolength{\textheight}{2cm}
\begin{document}
\title{Software code in R for performing instrumental variable analyses for Mendelian randomization investigations}
\author{maintained by Stephen Burgess}
\maketitle
This is a non-traditional publication to provide software code for the Mendelian randomization community in a single document. It will be updated when necessary as new methods are developed. Hopefully, this will become a collaborative resource than can be authored by the community rather than a single-author manuscript. However, I (Stephen Burgess, sb452@cam.ac.uk) retain the prerogative to exert editorial control.

Currently, it only contains R code. If someone wants to write Stata code or code for any other software package, this can also be included.

\clearpage

\tableofcontents % set depth

\clearpage

\section{Introduction and notation}
\begin{lstlisting}
### Dimensions
N # sample size
K # number of genetic variants

### Individual-level data
g # genetic variant(s), matrix dimension N x K
x # risk factor/exposure, vector length N
y # outcome, vector length N

###### Summary data
bx   # genetic associations with exposure, vector length K
by   # genetic associations with outcome, vector length K
bxse # standard errors of genetic associations with exposure
byse # standard errors of genetic associations with outcome
\end{lstlisting}

\clearpage

\section{Continuous outcome, individual-level data}
\subsection{Ratio (Wald) method -- single instrument (SNP or score)}
\begin{lstlisting}
bx   = lm(x~g)$coef[2]
bxse = summary(lm(x~g))$coef[2,2]
by   = lm(y~g)$coef[2]
byse = summary(lm(y~g))$coef[2,2]

# A. Ratio estimate

beta_ratio = lm(y~g)$coef[2]/lm(x~g)$coef[2]
beta_ratio = by/bx

# B. Asymptotic standard error (poor with weak instruments)

# 1. Delta method approximation (summarized data)

se_ratio_approx = summary(lm(y~g))$coef[2,2]/lm(x~g)$coef[2]
se_ratio_approx = byse/bx
         # first order approximation
se_ratio_approx = sqrt(summary(lm(y~g))$coef[2,2]^2/lm(x~g)$coef[2]^2 + lm(y~g)$coef[2]^2*summary(lm(x~g))$coef[2,2]^2/lm(x~g)$coef[2]^4 - 2*theta*lm(y~g)$coef[2]/lm(x~g)$coef[2]^3)
se_ratio_approx = sqrt(byse^2/bx^2 + by^2*bxse^2/bx^4 - 2*theta*by/bx^3)
         # second order approximation
         # theta is correlation between numerator
         #   and denominator in ratio estimate

# 2. Two-stage least squares method for standard error (individual-level data)

library(sem)
se_tsls = sqrt(tsls(y, cbind(x, rep(1,N)), cbind(g, rep(1,N)), w=rep(1,N))$V[1,1])


# C. Valid confidence intervals with weak instruments

# 1. Fieller's theorem
  f0 = by^2 - qt(0.975, N)^2 * byse^2
  f1 = bx^2 - qt(0.975, N)^2 * bxse^2
  f2 = by*bx
   D = f2^2 - f0*f1

  if(D>0) {
    r1 = (f2-sqrt(D))/f1
    r2 = (f2+sqrt(D))/f1
if(f1>0) { cat("Confidence interval is a closed interval [a,b]: \n a=", r1, ", b=", r2, sep="")) }
if(f1<0) { cat("Confidence interval is the union of two open intervals (-Inf, a], [b, +Inf): \n a=", r2, ", b=", r1, sep="")) }
          }
if(D<0|D==0) { cat("No finite confidence interval exists other than the entire real line.") }

# 2. Anderson--Rubin

library(ivpack)
ivmodel = ivreg(y~x|g, x=TRUE)
anderson.rubin.ci(ivmodel)
         # As with Fieller's theorem, interval may be a closed interval, the union of two open intervals, or undefined
\end{lstlisting}

\clearpage

\subsection{Two-stage least squares method}

\end{document}
